\section{Mateusz Matwiejuk}


\[c^2 = b^2 + a^2\]


super zdjecie jamnika \ref{fig:jamnik}
\begin{figure}[htbp]

    \centering
    \includegraphics[scale=0.4]{pictures/jamnik.png}
    
    \caption{jamnik}
    \label{fig:jamnik}
\end{figure}

\vspace{2 cm}

\textbf{tabela \ref{tab:mnoz}:}
\begin{table}[htbp]
\centering
\begin{tabular}{|l|l|l|l|l|l|}
\hline
           & \textbf{1} & \textbf{2} & \textbf{3} & \textbf{4} & \textbf{5} \\ \hline
\textbf{1} & 1          & 2          & 3          & 4          & 5          \\ \hline
\textbf{2} & 2          & 4          & 6          & 8          & 10         \\ \hline
\textbf{3} & 3          & 6          & 9          & 12         & 15         \\ \hline
\textbf{4} & 4          & 8          & 12         & 16         & 20         \\ \hline
\textbf{5} & 5          & 10         & 15         & 20         & 25         \\ \hline
\end{tabular}
\label{tab:mnoz}
\caption{Tabliczna mnozenia w zakresie 5 }
\end{table}

\vspace{2 cm}

lista nienumerowana:

\begin{itemize}
  \item element 1
  \item element 2
  \item element 3
\end{itemize}

\begin{itemize}
  \item[-] element 1
  \item[-] element 2
  \item[-] element 3
\end{itemize}


lista numerowana:
\begin{enumerate}
  \item element 1
  \item element 2
  \item element 3
\end{enumerate}
\vspace{2 cm}
\par
\begin{flushleft}
In humans, disease is often used more broadly to refer to any condition that causes pain, dysfunction, distress, social problems, or death to the person affected, or similar problems for those in contact with the person. In this broader sense, \emph{it sometimes includes injuries, disabilities, disorders, syndromes, infections, isolated symptoms, deviant behaviors, and atypical variations of structure and function, while in other contexts and for other purposes these may be considered distinguishable categories.}
\end{flushleft}
\begin{flushright}
\par
\underline{Death due to disease is called death by natural causes.} There are four main types of disease: infectious diseases, deficiency diseases, hereditary diseases (including both genetic and non-genetic hereditary diseases), and physiological diseases.
\end{flushright}

